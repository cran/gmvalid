\documentclass{article}
\usepackage[latin1]{inputenc}\usepackage[ae,hyper]{Rd}
\begin{document}
\HeaderA{dp}{Death penalty example of Simpson's paradox}{dp}
\keyword{datasets}{dp}
\begin{Description}\relax
Well known example to illustrate Simpson's paradox.
The data set shows that ignoring victim's race lead to a different conclusion
than including victim's race in the analysis.
\end{Description}
\begin{Usage}
\begin{verbatim}data(dp)\end{verbatim}
\end{Usage}
\begin{Format}\relax
A data frame with 326 observations on the following 3 variables.
\describe{
\item[\code{Defendants.Race}] a factor with levels \code{white} and \code{black}
\item[\code{Victims.Race}] a factor with levels \code{white} and \code{black}
\item[\code{Death.Penalty}] a factor with levels \code{yes} and \code{no}
}
\end{Format}
\begin{Source}\relax
Radelet ML (1981) \emph{Racial characteristics and the imposition of the Death penalty.}
American Sociological Review, 46(6):918-927.
\end{Source}
\begin{Examples}
\begin{ExampleCode}
data(dp)
## Graphical model analysis shows that 'defendants' race' is 
## independent from 'death penalty' given 'victims' race'.
gm.analysis(dp,program="coco",recursive=TRUE)
\end{ExampleCode}
\end{Examples}

\HeaderA{gm.analysis}{Analyze a data set}{gm.analysis}
\keyword{multivariate}{gm.analysis}
\keyword{nonparametric}{gm.analysis}
\keyword{graphs}{gm.analysis}
\keyword{hplot}{gm.analysis}
\keyword{models}{gm.analysis}
\begin{Description}\relax
Graphical model selection by different selection strategies. 
The selected graph is plotted where line thickness corresponds to an optional edge measure.
\end{Description}
\begin{Usage}
\begin{verbatim}
gm.analysis(data, edge.measure = c("gamma.cond", "marg.gamma", "cor", "boot",
            "cov", "p.value"), program = c("mim", "coco"),
            strategy = c("backwards", "forwards", "eh"), 
            plot.significant = TRUE, boot.N = 100, ...)
\end{verbatim}
\end{Usage}
\begin{Arguments}
\begin{ldescription}
\item[\code{data}] Data frame or a table (array) with at least 2 denominated variables. 
\item[\code{edge.measure}] One of "gamma.cond", "marg.gamma", "cor", "boot", "cov", "p.value". To be plotted in the graph. May be abbreviated. See details. 
\item[\code{program}] Selects whether MIM or the CoCo package is used for model selection. Calls either the function
\code{\LinkA{gm.coco}{gm.coco}} or \code{\LinkA{gm.mim}{gm.mim}}. May be abbreviated. 
\item[\code{strategy}] See \code{\LinkA{gm.coco}{gm.coco}} or \code{\LinkA{gm.mim}{gm.mim}}. 
\item[\code{plot.significant}] If TRUE only significant edges in the selected models are plotted (in solid lines). If FALSE also not significant edges are plotted as dashed lines.

\item[\code{boot.N}] Only required if \code{edge.measure} is "boot". Specifies the number of bootstrap replications. 
\item[\code{...}] Further options of \code{\LinkA{gm.coco}{gm.coco}} or \code{\LinkA{gm.mim}{gm.mim}} can be given. 
\end{ldescription}
\end{Arguments}
\begin{Details}\relax
Different edge measures:\\
"gamma.cond" - Conditional gamma coefficient, calls \code{\LinkA{gm.gamma}{gm.gamma}} with type = conditional.\\
"marg.gamma" - Marginal gamma coefficient, calls \code{\LinkA{gm.gamma}{gm.gamma}} with type = marginal.\\
"cor" - Pearson's correlation coefficient, calls \code{\LinkA{cor}{cor}}.\\
"boot" - calls \code{\LinkA{gm.boot.mim}{gm.boot.mim}} or \code{\LinkA{gm.boot.coco}{gm.boot.coco}} depending on the choice of \code{program}. \code{boot.N} bootstrap replications will be done. The relative edge frequency will be plotted next to it.\\
"cov" - Pearson's covariance, calls \code{\LinkA{cov}{cov}}. \\
"p.value" - is only available if \code{program} choice was "mim". Then \code{plot.significant} is set to TRUE since the p-value is only available for edges present in the model. In this case the edges will be plotted thinner the smaller the p-value.
\end{Details}
\begin{Value}
A list containing:
\begin{ldescription}
\item[\code{"strategy"}] One of "backwards","forwards" and "eh". 
\item[\code{"model" }] String vector of the accepted and plotted models. 
\item[\code{"edge.measure" }] Character string indicating what type of edge measure was performed. 
\item[\code{"analysis"}] Upper-tri matrix with the result of the edge measure values. 
\item[\code{"variable names"}] Matrix that assigns a letter to each variable that is used in the model formulas. 
\end{ldescription}
\end{Value}
\begin{Note}\relax
The option \code{boot} may take some time depending on the size of \code{boot.N}.
\end{Note}
\begin{Author}\relax
Fabian Sobotka, Marc Suling, Ronja Foraita \\
Bremen Institute for Prevention Research and Social Medicine \\
(BIPS)  \url{http://www.bips.uni-bremen.de}
\end{Author}
\begin{SeeAlso}\relax
\code{\LinkA{gm.plot}{gm.plot}}, \code{\LinkA{gm.coco}{gm.coco}}, \code{\LinkA{gm.mim}{gm.mim}}
\end{SeeAlso}
\begin{Examples}
\begin{ExampleCode}
  data(wam)
  gm.analysis(wam)
  
  ### showing various options in action
  gm.analysis(wam,program="c",strategy="f",edge.measure="b",
    boot.N=50,plot.significant=FALSE,recursive=TRUE,follow=TRUE,decomposable.mode=TRUE)
              
  gm.analysis(wam,edge.measure="p",options="u")
\end{ExampleCode}
\end{Examples}

\HeaderA{gm.boot.coco}{Graphical model validation using the bootstrap (CoCo).}{gm.boot.coco}
\keyword{nonparametric}{gm.boot.coco}
\keyword{multivariate}{gm.boot.coco}
\keyword{models}{gm.boot.coco}
\begin{Description}\relax
Validates a discrete undirected graphical model using the bootstrap.
To achieve that the CoCo package is used.
Relative frequencies of the bootstrapped models, cliques or edges are counted.
\end{Description}
\begin{Usage}
\begin{verbatim}
gm.boot.coco(N, data, strategy = c("backwards", "forwards", "eh"),
            calculations = c("subgraph", "diff", "edge", "clique"),
            model = FALSE, ...)
\end{verbatim}
\end{Usage}
\begin{Arguments}
\begin{ldescription}
\item[\code{N}] Number of bootstrap replications. 
\item[\code{data}] Data frame or a table (array). Variables should have names, \code{data} has to be discrete. 
\item[\code{strategy}] Type of model selection. "backwards" searches for not significant edges to delete, starting from the saturated model as default (\code{\LinkA{backward}{backward}}).
"forwards" adds significant edges, starting from the main effects model (\code{\LinkA{forward}{forward}}). The "Edwards-Havranek"
model search rejects complete models in every step and finishes with one or more accepted models (\code{\LinkA{eh}{eh}}).
The default strategy is "backwards". Selections may be abbreviated.

\item[\code{calculations}] String vector specifying the analysis methods. "subgraph", "clique" and "edge" show the frequency of occurence 
in all selected models of the bootstrap samples, while "diff" counts the differences
in the bootstrap replications compared to the edges selected from the original data set.
The frequency of the selected models of all bootstrap samples are always calculated. By default all calculations are done.
Selections may be abbreviated.

\item[\code{model}] Character string specifying a start model for "backwards" and "forwards" selection procedure.
For "eh" a character string would specify the base model for tests.
Since "eh" searches for models between a defined minimal and a maximal model, a string vector of length 2 is another option.
In this case a minimal and a maximal model is to be given.
The model formulae has to start with the first lowercase letters of the alphabet, e.g. "abc,cde". Variable names cannot be given.

\item[\code{...}] Additional options to be given to the selection procedures. See \code{\LinkA{backward}{backward}} (forward) or \code{\LinkA{eh}{eh}}. 
\end{ldescription}
\end{Arguments}
\begin{Details}\relax
This function uses a nonparametric bootstrap.\\
For your information about the advancements of the bootstrap, some run numbers will be displayed.
\end{Details}
\begin{Value}
A list containing:
\begin{ldescription}
\item[\code{"bootstrapped models" }] Relative frequency vector of selected models with model formulas as categories. 
\item[\code{"bootstrapped subgraphs" }] Relative frequencies vector of subgraphs that equals cliques or parts of a clique.
Returned if calculation "subgraph" is selected.

\item[\code{"bootstrapped cliques" }] Relative frequency vector of selected cliques. Returned if calculation "clique" is selected.

\item[\code{"bootstrapped edges"}] Matrix with relative frequencies of the edges. Returned if calculation "edge" is selected. 
\item[\code{"original model"}] Character string giving the selected model using the original unsampled data. Returned if calculation "diff" is selected. 
\item[\code{"differences from original data set"}] List with frequencies of more, less and different edges.
Sorted by occurence.
Returned if calculation "diff" is selected. 
\item[\code{"variable names"}] Matrix that assigns a letter to each variable that is used in the model formulas. 
\end{ldescription}
\end{Value}
\begin{Note}\relax
CoCo may crash R occasionally on a bad day. The R version of CoCo can only handle one CoCoObject.
You may run "endCoCo()" if you use CoCo also outside this package or the function aborts.\\ 
Running this function may take a while. A bootstrap with
data of size 10.000 times 8 and 1.000 replications may take the computer about 2 minutes.
\end{Note}
\begin{Author}\relax
Ronja Foraita, Fabian Sobotka \\
Bremen Institute for Prevention Research and Social Medicine \\
(BIPS)  \url{http://www.bips.uni-bremen.de}
\end{Author}
\begin{References}\relax
Efron B, Tibshirani RJ (1993) 
\emph{An Introduction to the Bootstrap.}
Chapman \& Hall

Badsberg JH (2001) 
\emph{A guide to CoCo.}
Journal of Statistical Software, 6(4).
\end{References}
\begin{SeeAlso}\relax
\code{\LinkA{gm.boot.mim}{gm.boot.mim}}
\end{SeeAlso}
\begin{Examples}
\begin{ExampleCode}
  ### should provide good results because of simulated data
  gm <- gm.modelsim(2000,"ABC,CDE")
  gm.boot.coco(50,gm,recursive=TRUE)
  
  ### on real data sets a forward bootstrap seems to have better results
  data(wynder)
  gm.boot.coco(100,wynder,strategy="f",calculations=c("s","e"),decomposable.mode=TRUE)
  
  ### with a given model
  data(wam)
  gm.boot.coco(10,wam,model="ab,bcde,cdef")
\end{ExampleCode}
\end{Examples}

\HeaderA{gm.boot.mim}{Graphical model validation using the bootstrap (MIM).}{gm.boot.mim}
\keyword{nonparametric}{gm.boot.mim}
\keyword{multivariate}{gm.boot.mim}
\keyword{models}{gm.boot.mim}
\begin{Description}\relax
Validates a discrete undirected graphical model using the bootstrap.
To achieve that the program MIM is used.
Relative frequencies of the bootstrapped models, cliques or edges are counted.
\end{Description}
\begin{Usage}
\begin{verbatim}
gm.boot.mim(N, data, strategy = c("backwards", "forwards", "eh"),
            calculations = c("subgraph", "diff", "edge", "clique"),
            model = FALSE, options = "")
\end{verbatim}
\end{Usage}
\begin{Arguments}
\begin{ldescription}
\item[\code{N}] Number of bootstrap replications. 
\item[\code{data}] Data frame or a table (array). Variables should have names, \code{data} has to be discrete. 
\item[\code{strategy}] Type of model selection. "backwards" searches for not significant edges to delete, starting from the saturated model as default.
"forwards" adds significant edges, starting from the main effects model. The "Edwards-Havranek"
model search rejects complete models in every step and finishes with one or more accepted models.
The default strategy is "backwards". Selections may be abbreviated.

\item[\code{calculations}] String vector specifying the analysis methods. "subgraph", "clique" and "edge" show the frequency of occurence 
in all selected models of the bootstrap samples, while "diff" counts the differences
in the bootstrap replications compared to the edges selected from the original data set.
The frequency of the selected models of all bootstrap samples are always calculated. By default all calculations are done.
Selections may be abbreviated.

\item[\code{model}] Character string specifying a start model for "backwards" and "forwards" selection procedure.
For the "eh" procedure a minimal and a maximal model has to be assigned in one string
connected with a " - " (see Example). The model formula has to start with the first lowercase letters of the alphabet, e.g. "abc,cde". Variable names cannot be given.

\item[\code{options}] Optional character string specifying further options for the search strategy. Possible options can be found in the MIM help searching for
"stepwise" (backwards, forwards) or "startsearch" (eh).  See \code{details}.

\end{ldescription}
\end{Arguments}
\begin{Details}\relax
This function uses a nonparametric bootstrap.\\
For your information about the advancements of the bootstrap, some run numbers will be displayed.

MIM options for stepwise procedures (backwards, forwards):\\
"A" - uses the AIC as selection criterion\\
"B" - uses the BIC as selection criterion\\
"J" - joggles between backward and forward\\
"N" - non coherent mode\\
"U" - unrestricted, allows for non-decomposable models;

MIM options for the eh modelsearch:\\
(positive number) - maximum number of models fitted\\
white space\\
(letters) - "U" for upward search, "D" for downward, default is both.
\end{Details}
\begin{Value}
A list containing:
\begin{ldescription}
\item[\code{"bootstrapped models" }] Relative frequency vector of selected models with model formulas as categories. 
\item[\code{"bootstrapped subgraphs" }] Relative frequencies vector of subgraphs that equals cliques or parts of a clique.
Returned if calculation "subgraph" is selected.

\item[\code{"bootstrapped cliques" }] Relative frequencies vector selected cliques. Returned if calculation "clique" is selected.

\item[\code{"bootstrapped edges"}] Matrix with relative frequencies of the edges. Returned if calculation "edge" is selected. 
\item[\code{"original model"}] Character string giving the selected model using the original unsampled data. Returned if calculation "diff" is selected. 
\item[\code{"differences from original data set"}] List with frequencies of more, less and different edges (See argument \code{calculations}.
Sorted by occurence.
Returned if calculation "diff" is selected. 
\item[\code{"variable names"}] Matrix that assigns a letter to each variable that is used in the model formulas. 
\end{ldescription}
\end{Value}
\begin{Note}\relax
The function requires the MIM program.
Make sure that it is running before using the function.\\ 
The package mimR will only work properly if your temporary directory has a path
where every folder has a name containing only 8 letters or less. mimR needs the Rgraphviz package. Therefore you will have to add "Bioconductor" to your R repositories.\\
Running this function may take a while. A bootstrap with
data of size 10.000 times 8 and 1.000 replications may take the computer about 4 minutes.
\end{Note}
\begin{Author}\relax
Ronja Foraita, Fabian Sobotka \\
Bremen Institute for Prevention Research and Social Medicine \\
(BIPS)  \url{http://www.bips.uni-bremen.de}
\end{Author}
\begin{References}\relax
Efron B, Tibshirani RJ (1993) 
\emph{An Introduction to the Bootstrap.}
Chapman \& Hall

Edwards D (2000) 
\emph{An Introduction to Graphical Modelling.}
Second Edition, Springer Verlag.
\end{References}
\begin{SeeAlso}\relax
\code{\LinkA{gm.boot.coco}{gm.boot.coco}}
\end{SeeAlso}
\begin{Examples}
\begin{ExampleCode}
  ### should provide good results because of simulated data
  gm.a <- gm.modelsim(2000,"ABC,CDE")
  gm.boot.mim(50,gm.a)
  
  ### on real data sets a forward bootstrap seems to have better results
  data(wynder)
  gm.boot.mim(100,wynder,strategy="f",calculations=c("s","e"),options="u")
  
  ### with model given
  data(wam)
  gm.boot.mim(10,wam,model="a,bcde,cdef")
  
  ### EH-strategy
  gm.boot.mim(50,wam,strategy="eh",model="a,bc,de,f - abcde,bcdef")
\end{ExampleCode}
\end{Examples}

\HeaderA{gm.coco}{Graphical model selection (CoCo)}{gm.coco}
\keyword{models}{gm.coco}
\keyword{multivariate}{gm.coco}
\begin{Description}\relax
Selects one or more graphical models (depending on strategy) using the CoCo package.
Different types of model selection are available. A base model can be optionally defined.
\end{Description}
\begin{Usage}
\begin{verbatim}
gm.coco(data, strategy = c("backwards", "forwards", "eh"), model = FALSE, ...)
\end{verbatim}
\end{Usage}
\begin{Arguments}
\begin{ldescription}
\item[\code{data}] Data frame or a table (array). Variables should have names, \code{data} has to be discrete. 
\item[\code{strategy}] Type of model selection. "backwards" searches for not significant edges to delete, starting from the saturated model as default (\code{\LinkA{backward}{backward}}).
"forwards" adds significant edges, starting from the main effects model (\code{\LinkA{forward}{forward}}). The "Edwards-Havranek"
model search rejects complete models in every step and finishes with one or more accepted models (\code{\LinkA{eh}{eh}}).
The default strategy is "backwards". Selections may be abbreviated.

\item[\code{model}] Character string specifying a start model for "backwards" and "forwards" selection procedure.
For "eh" a character string would specify the base model for tests.
Since "eh" searches for models between a defined minimal and a maximal model, a string vector of length 2 is another option.
In this case a minimal and a maximal model is to be given.
The model formulae has to start with the first lowercase letters of the alphabet, e.g. "abc,cde". Variable names cannot be given.

\item[\code{...}] Additional options to be given to the selection procedures. See \code{\LinkA{backward}{backward}} (forward) or \code{\LinkA{eh}{eh}}. 
\end{ldescription}
\end{Arguments}
\begin{Value}
A list containing:
\begin{ldescription}
\item[\code{"accepted" }] String vector of the accepted model(s). 
\item[\code{"rejected" }] String vector of the rejected model(s). 
\item[\code{"base"}] Character string giving the base model the selection started with. 
\item[\code{"strategy"}] Character string indicating what type of selection strategy was performed. 
\item[\code{"tests"}] Matrix containing the test results of the accepted vs. the rejected models 
(available if at least one of the models is submodel of another one).
Otherwise the accepted and rejected models are tested against the saturated model. 
\item[\code{"variable names"}] Matrix that assigns a letter to each variable used in the model formulas. 
\end{ldescription}
\end{Value}
\begin{Note}\relax
CoCo may crash R occasionally on a bad day. We tried to prevent all crash causes for discrete data. The R version of CoCo can only handle one CoCoObject.
You may run "endCoCo()" if you use CoCo also outside this package or the function aborts.
\end{Note}
\begin{Author}\relax
Ronja Foraita, Fabian Sobotka \\
Bremen Institute for Prevention Research and Social Medicine \\
(BIPS)  \url{http://www.bips.uni-bremen.de}
\end{Author}
\begin{References}\relax
Badsberg JH (2001)
\emph{A guide to CoCo.}
Journal of Statistical Software, 6(4).
\end{References}
\begin{SeeAlso}\relax
\code{\LinkA{gm.mim}{gm.mim}}, \code{\LinkA{backward}{backward}}, \code{\LinkA{eh}{eh}}
\end{SeeAlso}
\begin{Examples}
\begin{ExampleCode}
  data(wam)
  gm.coco(wam)
  ### giving many options to the strategy
  gm.coco(wam,recursive=TRUE,follow=TRUE,decomposable.mode=TRUE,
            coherent=TRUE,IC=TRUE,BIC=TRUE)
  
  ### giving base
  gm.coco(wam,strategy="e",model=c("ab,cd","ae,be"))
  gm.coco(wam,strategy="f",model="abc,cd,de,f")
\end{ExampleCode}
\end{Examples}

\HeaderA{gm.csi}{Conditional Synergy Index}{gm.csi}
\keyword{htest}{gm.csi}
\begin{Description}\relax
Computes the conditional synergy index of two dichotomous variables in relation to 
a binary group variable on the penetrance scale.
\end{Description}
\begin{Usage}
\begin{verbatim}
gm.csi(X, Y, group, data=0, reference=c(1,1,2), pen = NULL, conf.level = 0.95)
\end{verbatim}
\end{Usage}
\begin{Arguments}
\begin{ldescription}
\item[\code{X}] Index of the variable's position in \code{data} or a vector. 
\item[\code{Y}] Index of the variable's position in \code{data} or a vector. 
\item[\code{group}] Binary group or outcome variable addressed as variable index in 
\code{data} or as vector. By default, 2 is the reference category.
\item[\code{data}] Data frame or table. 
\item[\code{reference}] Vector to define the \code{reference} categories of \code{X}, \code{Y} and \code{group}. 
By default, 2 is the reference category for group.
\item[\code{pen}] If FALSE the penetrance P( group | X,Y ) is estimated from the data. 
\item[\code{conf.level}] Confidence level of the interval (default is 0.95). 
\end{ldescription}
\end{Arguments}
\begin{Details}\relax
Foraita's conditional synergy index (CSI) is an interaction measure on the penetrance scale
between two dichotomous variables on a  binary outcome. 
The index equals 1 under additivity, CSI > 1 in the case of antagonism and 
CSI < 1 in the presence of synergy.

The confidence intervals are calculated using an asymptotic variance given in Foraita (2007).
\end{Details}
\begin{Value}
A list containing:
\begin{ldescription}
\item[\code{penetrance.ratio}] a penetrance ratio table of P( group=reference | X,Y ) / P( group=NOT reference| X,Y ) 
\item[\code{measure}] a matrix containing the estimate, standard deviation, 
confidence interval and p-value. Figures in brackets show the
reference category respectively the category under consideration.
\end{ldescription}
\end{Value}
\begin{Author}\relax
Ronja Foraita, Fabian Sobotka \\
Bremen Institute for Prevention Research and Social Medicine \\
(BIPS)  \url{http://www.bips.uni-bremen.de}
\end{Author}
\begin{References}\relax
Foraita R (2007)
\emph{A conditional synergy index to assess biological interaction.}
Working Paper. Please send an e-mail to \email{foraita@bips.uni-bremen.de}.
\end{References}
\begin{SeeAlso}\relax
\code{\LinkA{gm.si}{gm.si}}
\end{SeeAlso}
\begin{Examples}
\begin{ExampleCode}
  data(idd35)
  gm.csi(1,2,3,data=idd35)

  ### >> constructing an additive and multiplicative penetrance
  x <- c(0.1,0.4)
  y <- c(0.05,0.5)
  add.pen <- outer(x,y,FUN="+")
  mult.pen <- outer(x,y)
  het.pen <- outer(x,y,FUN="+") - outer(x,y)

  ### >> Function that samples data using the penetrance 
  make.data <- function(R,pen,category) 
    {
      s.vec <- sample(c(1,2,3,4),R,replace=TRUE,prob=as.vector(pen))
      fact.1 <- fact.2 <- vector()
      for( i in 1:R ) {
        ifelse( s.vec[i] == 1 || s.vec[i] == 3 , fact.1[i] <- 1, fact.1[i] <- 2 ) 
        ifelse( s.vec[i] == 1 || s.vec[i] == 2 , fact.2[i] <- 1, fact.2[i] <- 2 ) 
      }
      cbind(X=fact.1,Y=fact.2,group=rep(category,R))  
    }

  ### >>> Building datasets with affected and unaffected subjects   
  add.aff <- make.data(200,add.pen,2)
  add.uaf <- make.data(200,1-add.pen,1)  
  add.df <- as.data.frame(rbind(add.uaf,add.aff))
  
  mult.aff <- make.data(200,mult.pen,2)
  mult.uaf <- make.data(200,1-mult.pen,1)  
  mult.df <- as.data.frame(rbind(mult.uaf,mult.aff))
  
  het.aff <- make.data(200,het.pen,2)
  het.uaf <- make.data(200,1-het.pen,1)  
  het.df <- as.data.frame(rbind(het.uaf,het.aff))
   
  gm.csi(1,2,3,add.df,pen=add.pen)   # Additivity
  gm.csi(1,2,3,mult.df,pen=mult.pen) # Synergy
  gm.csi(1,2,3,het.df,pen=het.pen)   # Antagonism
\end{ExampleCode}
\end{Examples}

\HeaderA{gm.cv}{Cross Validation for Graphical (Chain) Models}{gm.cv}
\keyword{models}{gm.cv}
\keyword{regression}{gm.cv}
\begin{Description}\relax
Predicts a binary outcome variable in a given graphical (chain) model
using k-fold cross validation.
\end{Description}
\begin{Usage}
\begin{verbatim}
gm.cv(k, data, outcome=1, strategy = c("backwards", "forwards"),
        chain = FALSE, options="", conf.level = 0.95)
\end{verbatim}
\end{Usage}
\begin{Arguments}
\begin{ldescription}
\item[\code{k}] Number of folds the data should be split into in order to estimate the success of prediction in the cross validation. 
\item[\code{data}] Data frame or a table (array). Variables should have names, \code{data} has to be discrete. 
\item[\code{outcome}] Variable index of the outcome variable. Default is 1.
\item[\code{strategy}] Type of model selection. "backwards" searches for not significant edges to delete, starting from the saturated model as default.
"forwards" adds significant edges, starting from the main effects model. 
The default strategy is "backwards". Selections may be abbreviated.

\item[\code{chain}] Character string to specify the block structure of directed graphical models.
The syntax is "vs1|vs2|vs3" where vs1,vs2,vs3 are sets of variables and the variables in vs1 are prior to those in vs2 etc. Otherwise the model will be undirected. 
Only lowercase letters are allowed!
\item[\code{options}] Character string specifying further options for the search strategy. Possible options can be found in the MIM help when searching for
"stepwise" (backwards, forwards) or "startsearch" (eh).  See \code{details}.

\item[\code{conf.level}] Confidence level of the interval (default is 0.95). 
\end{ldescription}
\end{Arguments}
\begin{Details}\relax
Outcome variable must be the first in \code{data}, categorized as 
1="unaffected" and 2="affected".

The procedure is as follows

1. Data Processing:\\
The data set is splitted into k folds

2. Model Selection:\\
A model is selected using the observations from k - j-th folds, j=1,...,k.
P-values of edges are stored. 

3. Calculate Risk:\\
A ratio table of being affected rather than unaffected is calulated for the joint probability of all influences associated with the outcome variable.
This is done for every fold.\\    
If the ratio is greater or equal to one, the risk is set to 2, otherwise it is set to 1. For each fold a risk table is generated.

4. Prediction:\\
The risk table from step(3), corresponding to the k-j-th fold, is used to predict the observations in the j-th fold.
The prediction "PRED" is compared to the real outcome "OUT" in each fold by calculating the success probability using the following formula:

success.prob = 1/n ( 1 - SUM(1:n)|OUT - PRED|),\\
n = number of oberservations in the j-th fold \\

MIM options for stepwise procedures (backwards, forwards):\\
"A" - uses the AIC as selection criterion\\
"B" - uses the BIC as selection criterion\\
"J" - joggles between backward and forward\\
"N" - non coherent mode\\
"U" - unrestricted, allows for non-decomposable models;
\end{Details}
\begin{Value}
A list containing:
\begin{ldescription}
\item[\code{ pvalue }] Matrix of the calculated p-values in each fold for each edge obtained by the model selection. 
NA's mark missing edges. 
\item[\code{ ratio }] List of ratio tables (see details, step (3)). 
\item[\code{ risk }] List of risk tables (see details, step (3)). 
\item[\code{ success }] Matrix with the best prediction models in each fold, the number of edges that point
to the outcome variable and the probability of successful prediction (see details, step (4)). 
The initial blockstructure in the prediction step is: "variable set of influences | outcome variable".   
A given \code{chain} is only used during the model selection step.                   
The success probability is calculated using those clique structures the outcome variable is involved in.

\end{ldescription}
\end{Value}
\begin{Note}\relax
The function requires the MIM program.
Make sure that it is running before using the function.
mimR will only work properly if your temporary directory has a path
where every folder has a name containing only 8 letters or less. mimR needs the Rgraphviz package.
Therefore you will have to add "Bioconductor" to your R repositories.
\end{Note}
\begin{Author}\relax
Ronja Foraita, Fabian Sobotka \\
Bremen Institute for Prevention Research and Social Medicine \\
(BIPS)  \url{http://www.bips.uni-bremen.de}
\end{Author}
\begin{References}\relax
Foraita R (2008) 
\emph{Outcome prediction in graphical (chain) models using cross validation.}
Slides. Please contact \email{foraita@bips.uni-bremen.de}.

Edwards D (2000) 
\emph{An Introduction to Graphical Modelling.}
Second Edition, Springer Verlag.
\end{References}
\begin{SeeAlso}\relax
\code{\LinkA{gm.mim}{gm.mim}}
\end{SeeAlso}
\begin{Examples}
\begin{ExampleCode}
  
  ABC <- gm.modelsim(500,"ABC,CD")
  out <- gm.cv(5,data=ABC, strategy="f")
  out
  
  ### DAG using a stepwise selection
  out.dag <- gm.cv(3,data=ABC,option="j",chain="d|b|c|a")  
  
  ### Chain graph using BIC as selection criteria and allowing for 
  ### non-decomposable models
  cg <- gm.modelsim(1000,"ABD,BCE")  
  out.cg <- gm.cv(3,data=cg,option="bu",chain="cb|de|a")
  
  ## Not run: 
gm.cv(3,data=ABC,chain="DBD|A") # you have to use lowercase letters
            gm.cv(3,data=ABC,chain="dca|b") # a is supposed to be outcome variable 
                                  # and thus have to be in the very right block    
            
## End(Not run)
\end{ExampleCode}
\end{Examples}

\HeaderA{gm.gamma}{Conditional Gamma coefficient estimation and confidence intervals}{gm.gamma}
\keyword{htest}{gm.gamma}
\begin{Description}\relax
Calculates conditional gamma coefficients of two binary or ordinal variables X and Y given a 
set of variables \{A,B,...\}.
\end{Description}
\begin{Usage}
\begin{verbatim}
  gm.gamma(X = 0, Y = 0, data = 0, conditions = 0,
        type = c("conditional", "single", "marginal"), conf.level = 0.95)
\end{verbatim}
\end{Usage}
\begin{Arguments}
\begin{ldescription}
\item[\code{X}] Index of the variable's position in \code{data} or a vector. 
\item[\code{Y}] Index of the variable's position in \code{data} or a vector. 
\item[\code{data}] Data frame or table 
\item[\code{conditions}] Variable indices in \code{data} or a data frame of the conditioning variables. 
\item[\code{type}] Character string specifying the method,
must be one of "conditional" (default), "single" or "marginal". 
May be abbreviated. 
\item[\code{conf.level}] Confidence level of the interval (default is 0.95). 
\end{ldescription}
\end{Arguments}
\begin{Details}\relax
The gamma coefficient is a correlation measure between two
binary or ordinal variables. It ranges between -1 and 1, where -1 or respectively
+1 stands for a purely negative or positive monotone relation. The relation has
not to be of linear nature!

If \code{type} = "conditional", the conditional gamma coefficients are calculated and
if \code{type} = "marginal", the marginal gamma coefficients are computed. \\
If \code{X} and \code{Y} are given, the "single" gamma coefficient between both
variables are computed.\\
If \code{X} or \code{Y} are zero, the function computes all possible conditional gamma coefficients.

Confidence intervals are calculated using the asymptotic variance given in Olszak and Ritschard (1995).
\end{Details}
\begin{Value}
A matrix containing the gamma estimate(s), standard deviation(s), confidence interval(s) and p-value(s).
\end{Value}
\begin{Author}\relax
Ronja Foraita, Fabian Sobotka \\
Bremen Institute for Prevention Research and Social Medicine \\
(BIPS)  \url{http://www.bips.uni-bremen.de}
\end{Author}
\begin{References}\relax
Davis JA (1967) 
\emph{A partial coefficient for Goodman and Kruskal's gamma.}
Journal of the American Statistical Association, 62:189-193.

Olszak M, Ritschard G (1995) 
\emph{The behaviour of nominal and ordinal partial association measures.}
The Statistician, 44(2):195-212.
\end{References}
\begin{SeeAlso}\relax
\code{\LinkA{gm.or}{gm.or}}, \code{\LinkA{gm.rr}{gm.rr}}
\end{SeeAlso}
\begin{Examples}
\begin{ExampleCode}
  data(dp)

  ### Conditional Gamma by victime
  gm.gamma(1,3,conditions=2,data=dp)
  ### the same
  gm.gamma(dp$Defendants.Race,dp$Death.Penalty,data=dp,conditions=dp$Victims.Race)
  
  ### Stratified Gamma
  dp.black <- data.frame(victime=dp$Victims.Race[dp$Victims.Race=="black"],
                        killer=dp$Defendants.Race[dp$Victims.Race=="black"],
                        death.penalty=dp$Death.Penalty[dp$Victims.Race=="black"])
  dp.white <- data.frame(victime=dp$Victims.Race[dp$Victims.Race=="white"],
                        killer=dp$Defendants.Race[dp$Victims.Race=="white"],
                        death.penalty=dp$Death.Penalty[dp$Victims.Race=="white"])  
  table(dp.black[,c(2,3,1)])
  table(dp.white[,c(2,3,1)])  

  gm.gamma(2,3,data=dp.black)  
  gm.gamma(2,3,data=dp.white)  
  
  ### Marginal Gamma
  gm.gamma(1,3,data=dp)

  ### Analyse complete data set
  gm.gamma(data=dp,type="m")
  
  ### Plot model
  gamma <- gm.gamma(data=dp)
   #> all edges
  mat <- matrix(NA,nrow=3,ncol=3)
  mat[upper.tri(mat)] <- gamma[,1]
  gm.plot(model="abc",data.analysis=mat)
   #> only significant edges
  mat <- matrix(NA,nrow=3,ncol=3)   
  tmp <- vector()
  for( i in 1:dim(gamma)[1] ) ifelse(gamma[i,5]<0.05, tmp[i] <- gamma[i,1], tmp[i] <-NA)
  mat[upper.tri(mat)] <- tmp
  gm.plot(model="ab,bc",data.analysis=mat)
\end{ExampleCode}
\end{Examples}

\HeaderA{gm.generate}{Random data frames of binary variables with given marginals}{gm.generate}
\keyword{datagen}{gm.generate}
\keyword{distribution}{gm.generate}
\begin{Description}\relax
Generates a random data frame of independent binary variables with given marginals.
\end{Description}
\begin{Usage}
\begin{verbatim}
gm.generate(N, p = c(0.5, 0.5, 0.5))
\end{verbatim}
\end{Usage}
\begin{Arguments}
\begin{ldescription}
\item[\code{N}] Number of observations, sample size. 
\item[\code{p}] Probability vector. Each element assigns the probability to draw a "2". The user-defined number of
elements determines the number of variables in the resulting data frame. 
\end{ldescription}
\end{Arguments}
\begin{Value}
A data frame with number of rows equal to \code{N} and number of columns equal to the length of \code{p}.
\end{Value}
\begin{Author}\relax
Ronja Foraita, Fabian Sobotka \\
Bremen Institute for Prevention Research and Social Medicine \\
(BIPS)  \url{http://www.bips.uni-bremen.de}
\end{Author}
\begin{SeeAlso}\relax
\code{\LinkA{gm.modelsim}{gm.modelsim}}, \code{\LinkA{gm.sim.ixj}{gm.sim.ixj}}, \code{\LinkA{r2dtable}{r2dtable}}
\end{SeeAlso}
\begin{Examples}
\begin{ExampleCode}
gm.generate(10,c(.5,.2,.2))
gm.generate(15,c(.5,.5,.5,.5,.5,.5))
\end{ExampleCode}
\end{Examples}

\HeaderA{gm.mim}{Graphical model selection (MIM)}{gm.mim}
\keyword{models}{gm.mim}
\keyword{multivariate}{gm.mim}
\begin{Description}\relax
Selects one or more graphical models (depending on strategy) using the program MIM.
Different types of model selection are available. A base model can be optionally defined.
\end{Description}
\begin{Usage}
\begin{verbatim}
gm.mim(data, strategy = c("backwards", "forwards", "eh"),
        model = FALSE, chain = FALSE, options = "")
\end{verbatim}
\end{Usage}
\begin{Arguments}
\begin{ldescription}
\item[\code{data}] Data frame or a table (array). Variables should have names. \code{data} has to be discrete. 
\item[\code{strategy}] Type of model selection. "backwards" searches for not significant edges to delete, starting from the saturated model as default.
"forwards" adds significant edges, starting from the main effects model. The "Edwards-Havranek"
model search rejects complete models in every step and finishes with one or more accepted models.
The default strategy is "backwards". Selections may be abbreviated.

\item[\code{model}] Character string specifying a start model for "backwards" and "forwards" selection procedure.
For the "eh" procedure a minimal and a maximal model has to be assigned in one string
connected with a " - " (see Example). The model formulae has to start with the first lowercase letters of the alphabet, e.g. "abc,cde".
Variable names cannot be given.

\item[\code{chain}] Character string to specify the block structure of directed graphical models.
The syntax is "vs1|vs2|vs3" where vs1,vs2,vs3 are sets of variables and the variables in vs1 are prior to those in vs2 etc.
The formula has to be given with the first lowercase letters of the alphabet, e.g. "a|bc|d". Variable names cannot be given.
For more information search the MIM help for "block overview".

\item[\code{options}] Character string specifying further options for the search strategy. Possible options can be found in the MIM help when searching for
"stepwise" (backwards, forwards) or "startsearch" (eh).  See \code{details}.

\end{ldescription}
\end{Arguments}
\begin{Details}\relax
MIM options for stepwise procedures (backwards, forwards):\\
"A" - uses the AIC as selection criterion\\
"B" - uses the BIC as selection criterion\\
"J" - joggles between backward and forward\\
"N" - non coherent mode\\
"U" - unrestricted, allows for non-decomposable models;

MIM options for the eh modelsearch:\\
(positive number) - maximum number of models fitted\\
white space\\
(letters) - "U" for upward search, "D" for downward, default is both.
\end{Details}
\begin{Value}
A list containing:
\begin{ldescription}
\item[\code{"accepted" }] String vector of the accepted model(s). 
\item[\code{"rejected" }] String vector of the rejected model(s). 
\item[\code{"base"}] Character string with the base model the selection started with. NA for "eh". 
\item[\code{"strategy"}] Character string indicating what type of selection strategy was performed. 
\item[\code{"tests against saturated"}] Matrix giving the test results of the accepted and the rejected models against the saturated model.

\item[\code{"p values"}] Upper-tri matrix with p-values of those edges in the accepted models. NA if edge is misseing. 
\item[\code{"variable names"}] Matrix that assigns a letter to each variable that is used in the model formulas. 
\end{ldescription}
\end{Value}
\begin{Note}\relax
The function requires the MIM program.
Make sure that it is running before using the function. The package mimR will only work properly if your temporary directory has a path
where every folder has a name containing only 8 letters or less. mimR needs the Rgraphviz package.
Therefore you will have to add "Bioconductor" to your R repositories.
\end{Note}
\begin{Author}\relax
Ronja Foraita, Fabian Sobotka \\
Bremen Institute for Prevention Research and Social Medicine \\
(BIPS)  \url{http://www.bips.uni-bremen.de}
\end{Author}
\begin{References}\relax
Edwards D (2000) 
\emph{An Introduction to Graphical Modelling.}
Second Edition, Springer Verlag.
\end{References}
\begin{SeeAlso}\relax
\code{\LinkA{gm.coco}{gm.coco}}
\end{SeeAlso}
\begin{Examples}
\begin{ExampleCode}
  data(wam)
  gm.mim(wam)
  ### giving strategy
  gm.mim(wam,strategy="e")
  
  ### giving minimal and maximal model
  gm.mim(wam,strategy="e",model="a,bc,de,f - abcd,cdef")
  ### giving block structure
  gm.mim(wam,strategy="f",model="a,b,c|abc,de|abcd,ef",chain="abc|de|f",options="BNU")
\end{ExampleCode}
\end{Examples}

\HeaderA{gm.modelsim}{Random data frames with given dependence model and marginals}{gm.modelsim}
\keyword{datagen}{gm.modelsim}
\keyword{graphs}{gm.modelsim}
\begin{Description}\relax
Generates a random data frame of discrete variables with a given dependence model and marginals.
\end{Description}
\begin{Usage}
\begin{verbatim}
    gm.modelsim(N, model, categories = 0)
\end{verbatim}
\end{Usage}
\begin{Arguments}
\begin{ldescription}
\item[\code{N}] Number of observations, sample size. 
\item[\code{model}] A character string assigning a dependence model expressed as clique structure. Each variable
has to be expressed as a letter, e.g. "ABC,CDE".

\item[\code{categories}] a list of weight vectors that assigns the weight of each catogory.
Number of list elements must equal the number of variables in \code{model}.
Default is "list(c(.5,.5),c(.5,.5),...)".

\end{ldescription}
\end{Arguments}
\begin{Value}
A data frame with number of rows approximately equal to \code{N} and number of columns equal
to the number of variables used in \code{model}.
\end{Value}
\begin{Note}\relax
Observed marginal probabilities reflect the given marginal probabilites only approximatively.
Works best with population sizes over N=10,000.
\end{Note}
\begin{Author}\relax
Ronja Foraita, Fabian Sobotka \\
Bremen Institute for Prevention Research and Social Medicine \\
(BIPS)  \url{http://www.bips.uni-bremen.de}
\end{Author}
\begin{SeeAlso}\relax
\code{\LinkA{gm.generate}{gm.generate}}, \code{\LinkA{gm.sim.ixj}{gm.sim.ixj}}, \code{\LinkA{r2dtable}{r2dtable}}
\end{SeeAlso}
\begin{Examples}
\begin{ExampleCode}
    gm.modelsim(100,"AB,AC")
    table( gm.modelsim(100,"a,b,c") )
    
    tmp.df <- gm.modelsim(10000,"abf,cd,cf,bdeg,bfg")
    
    # with given number of categories
    tmp.df <- gm.modelsim(1000,"AB,C",list(c(1,1,1),c(1,1),c(1,1,1)))

    # with given number of categories and marginals
    tmp.df <- gm.modelsim(1000,"ABC",list(c(0.3,0.3,0.4),c(0.6,.4),c(0.25,0.25,0.5)))
    table(tmp.df)

    ## Not run: 
tmp.df <- gm.modelsim(100,"ABC",list(3,2,3))# (number of categories will be 2 x 2 x 2 )
            gm.modelsim(100,"123")
            
## End(Not run)
\end{ExampleCode}
\end{Examples}

\HeaderA{gm.or, gm.rr}{Stratified odds ratios or risk ratios}{gm.or, gm.rr}
\aliasA{gm.or}{gm.or, gm.rr}{gm.or}
\aliasA{gm.rr}{gm.or, gm.rr}{gm.rr}
\keyword{htest}{gm.or, gm.rr}
\begin{Description}\relax
Computes (stratified) odds ratios or risk ratios and their asymptotic confidence interval.
\end{Description}
\begin{Usage}
\begin{verbatim}
gm.or(X, Y, data = 0, conditions = 0,
        reference = c("last", "biggest", "first"), conf.level = 0.95)

gm.rr(X, Y, data = 0, conditions = 0,
        reference = c("last", "biggest", "first"), conf.level = 0.95)
\end{verbatim}
\end{Usage}
\begin{Arguments}
\begin{ldescription}
\item[\code{X}] Index of the variable's position in \code{data} or a vector. 
\item[\code{Y}] Index of the variable's position in \code{data} or a vector. 
\item[\code{data}] Data frame or a table 
\item[\code{conditions}] Variable indices in \code{data} or a data frame of the conditioning variables. 
\item[\code{reference}] Character string specifying the reference category, 
must be one of "last" (default), "biggest" (largest) or "first". 
May be abbreviated. 
\item[\code{conf.level}] confidence level of the interval (default is 0.95). 
\end{ldescription}
\end{Arguments}
\begin{Details}\relax
Calculates odds ratios by conditional maximum likelihood estimation (Fisher) for stratified odds ratios and
odds ratios by unconditional maximum likelihood estimation (Wald) for marginal odds ratios. 
Confidence intervals are calculated using exact methods.

Calculates risk ratios by unconditional maximum likelihood estimation (Wald). 
Confidence intervals are calculated using normal approximation.

Is based on the functions \code{oddsratio.fisher}, \code{oddsratio.wald} and \code{riskratio.wald} (package: epitools).
\end{Details}
\begin{Value}
A matrix containing the estimate(s), confidence interval(s) and p-value(s).
\end{Value}
\begin{Author}\relax
Ronja Foraita, Fabian Sobotka \\
Bremen Institute for Prevention Research and Social Medicine \\
(BIPS)  \url{http://www.bips.uni-bremen.de}
\end{Author}
\begin{References}\relax
Rothman KJ, Greenland S (1998) 
\emph{Modern Epidemiology.} (2nd)
Lippincott-Raven Publisher

\url{http://www.epitools.net}
\end{References}
\begin{SeeAlso}\relax
\code{\LinkA{oddsratio}{oddsratio}}, \code{\LinkA{riskratio}{riskratio}}
\end{SeeAlso}
\begin{Examples}
\begin{ExampleCode}
  group  <- c("treatment","placebo1","placebo2")
  target <- c("low","medium","high")
  mat    <- matrix(c(78,35,53,77,10,89,16,119,32),nrow=3,ncol=3,byrow=TRUE, 
                    dimnames=list("group"=group,"target"=target))
  treat  <- data.frame(expand.table(mat))
  table(treat)
  
  ### Marginal OR
  gm.or(1,2,treat,reference="f")
  gm.or(treat$target,treat$group)
  
  ### Stratified OR
  data <- gm.modelsim(1000,"ab,bcd",list(c(1,1),c(1,1),c(1,1),c(1,1)))  
  gm.or(1,2,conditions=c(3,4),data=data)
  
  ### Marginal RR
  gm.rr(1,2,treat,reference="f")
  gm.rr(treat$target,treat$group)
  
  ### Stratified RR
  data <- gm.modelsim(1000,"ab,bcd",list(c(1,1),c(1,1),c(1,1),c(1,1)))  
  gm.rr(1,2,conditions=c(3,4),data=data)
  
  ### ALSO
  gm.or(X=data$a,Y=data$b,conditions=data$d)

\end{ExampleCode}
\end{Examples}

\HeaderA{gm.plot}{Plot graphical models}{gm.plot}
\keyword{hplot}{gm.plot}
\keyword{graphs}{gm.plot}
\begin{Description}\relax
Plots given graphical models and writes provided measures next to the edges.
\end{Description}
\begin{Usage}
\begin{verbatim}
gm.plot(model, significant = TRUE, data.analysis = 0)
\end{verbatim}
\end{Usage}
\begin{Arguments}
\begin{ldescription}
\item[\code{model}] String vector with model formulas. See \code{\LinkA{gm.modelsim}{gm.modelsim}}. 
\item[\code{significant}] If TRUE only significant edges in the selected models are plotted (in solid lines). If FALSE also not significant edges are plotted as dashed lines.

\item[\code{data.analysis}] Upper-tri matrix with the measure for the edge between variables i and j (i>j) at matrix position [i,j]. If the length of the \code{model} is bigger than 1, \code{data.analysis} has to be a list of matrices. 
\end{ldescription}
\end{Arguments}
\begin{Details}\relax
The line width of the edges will depend on the size of the numbers in \code{data.analysis}.
\end{Details}
\begin{Value}
TRUE
\end{Value}
\begin{Note}\relax
Every use of the plot function opens a new window.
\end{Note}
\begin{Author}\relax
Fabian Sobotka, Marc Suling, Ronja Foraita \\
Bremen Institute for Prevention Research and Social Medicine \\
(BIPS)  \url{http://www.bips.uni-bremen.de}
\end{Author}
\begin{SeeAlso}\relax
\code{\LinkA{gm.analysis}{gm.analysis}}
\end{SeeAlso}
\begin{Examples}
\begin{ExampleCode}
  gm.plot("ABC,CDE")

  gm.plot("VBA,EVC")
  
  gm.plot(c("ABC,CDE","AB,BC,CD,DE","ABC,DEF,GHI"))
  
  gm.plot("AB,AC",FALSE,matrix(0.5,nrow=3,ncol=3))
\end{ExampleCode}
\end{Examples}

\HeaderA{gm.si}{Synergy Index}{gm.si}
\keyword{htest}{gm.si}
\begin{Description}\relax
Computes the synergy index of two discrete variables in relation to 
a binary group variable.
\end{Description}
\begin{Usage}
\begin{verbatim}
gm.si(X,Y,group,data=0,reference=c(1,1,2),conf.level = 0.95)
\end{verbatim}
\end{Usage}
\begin{Arguments}
\begin{ldescription}
\item[\code{X}] Index of the variable's position in \code{data} or a vector. 
\item[\code{Y}] Index of the variable's position in \code{data} or a vector. 
\item[\code{group}] Binary group or outcome variable addressed as variable index in 
\code{data} or as vector.
\item[\code{data}] Data frame or table. 
\item[\code{reference}] Vector to define the \code{reference} categories of \code{X}, \code{Y} and \code{group}.
By default, 2 is the reference category for group.
\item[\code{conf.level}] Confidence level of the interval (default is 0.95). 
\end{ldescription}
\end{Arguments}
\begin{Details}\relax
Rothman's synergy index (S) is an interaction measure between two
discrete variables on a dichotomous outcome. 
The index equals 1 under additivity, S > 1 in the case of synergy and 
S < 1 in the presence of antagonism.

The synergy index is originally constructed on (2 x 2)-tables, but (i x j)-tables can
also be analyzed. Then (i-1) x (j-1) synergy indices are computed and combined to 
an overall synergy index.    

The confidence intervalls are calculated using the asymptotic variance given in 
Rothman (1974).
\end{Details}
\begin{Value}
A list containing:
\begin{ldescription}
\item[\code{ratio }] A rate ratio table, more precise [P( group=reference,X,Y ) / P( group=NOT reference,X,Y )] / [min(P( group=reference,X,Y ) / P( group=NOT reference,X,Y ))] 
\item[\code{covariance}] Covariances matrix of the single synergy indices. 
Not written if \code{X} or \code{Y} are binary.
\item[\code{measure }] Matrix containing the estimate(s), standard deviation(s), 
confidence interval(s) and p-value(s). Figures in brackets show the
reference category respectively the category under consideration.\\                 
If both factors \code{X} and \code{Y} are binary, confidence 
intervals for case-control as well as cohort designs are computed.
If at least one factor has more than 2 categories, the overall synergy index
with its corresponding confidence interval is computed that follows a 
case-control design.
\end{ldescription}
\end{Value}
\begin{Note}\relax
It can occur that certain combinations of categories lead to a negative synergy indices.
In that case no confidence intervals can be computed. If so, use the \code{reference} option
to re-order the categories of the variable(s) in question (see example below).
\end{Note}
\begin{Author}\relax
Ronja Foraita, Fabian Sobotka \\
Bremen Institute for Prevention Research and Social Medicine \\
(BIPS)  \url{http://www.bips.uni-bremen.de}
\end{Author}
\begin{References}\relax
Rothman K (1974) 
\emph{The estimation of synergy or antagonism.}
American Journal of Epidemiology, 103(5):506-511

Rothman K (1986) 
\emph{Modern Epidemiology.}
Little, Brown and Company, Boston/Toronto.
\end{References}
\begin{SeeAlso}\relax
\code{\LinkA{gm.csi}{gm.csi}}
\end{SeeAlso}
\begin{Examples}
\begin{ExampleCode}
  data(wynder)
  gm.si(1,2,3,wynder)

  # Smoking and alcohol in relation to oral cancer among male veterans under age 60.
  # (from "Modern Epidemiology")
  oral <- array(c(20,3,18,8,12,6,166,225),dim=c(2,2,2), 
            dimnames=list(Group=c("control","cases"),
            Smoker=c("no","yes"),Alcohol=c("no","yes")))
  oral.df <- expand.table(oral)
  # grouping variable is first in data frame
  gm.si(2,3,1,oral.df)
  
  # Effects must be ascending in respect to the reference category
  show.effect <- array(c(1,7,2,7,7,12,106,48),dim=c(2,2,2),
                        dimnames=list(A=1:2,B=1:2,C=1:2))
  # produces NaN
  gm.si(1,2,3,expand.table(show.effect))
  # > re-ordering variable B helps
  gm.si(1,2,3,expand.table(show.effect),reference=c(1,2,2))

\end{ExampleCode}
\end{Examples}

\HeaderA{gm.sim.ixj}{Random (i x j)-way dependency tables with given marginals}{gm.sim.ixj}
\keyword{datagen}{gm.sim.ixj}
\begin{Description}\relax
Generates random (i x j)-way dependency tables with given marginals.
\end{Description}
\begin{Usage}
\begin{verbatim}
gm.sim.ixj(N, pa, pb)
\end{verbatim}
\end{Usage}
\begin{Arguments}
\begin{ldescription}
\item[\code{N}] Number of observations; sample size. 
\item[\code{pa}] Weight vector of length i to assign the distribution of the row totals. 
\item[\code{pb}] Weight vector of length j to assign the distribution of the column totals. 
\end{ldescription}
\end{Arguments}
\begin{Value}
A (i x j)-matrix.
\end{Value}
\begin{Note}\relax
Observed totals that were generated according to the shorter weight vector 
reflect the given weights only approximatively.
\end{Note}
\begin{Author}\relax
Ronja Foraita, Fabian Sobotka \\
Bremen Institute for Prevention Research and Social Medicine \\
(BIPS)  \url{http://www.bips.uni-bremen.de}
\end{Author}
\begin{SeeAlso}\relax
\code{\LinkA{gm.generate}{gm.generate}}, \code{\LinkA{gm.modelsim}{gm.modelsim}}, \code{\LinkA{r2dtable}{r2dtable}}
\end{SeeAlso}
\begin{Examples}
\begin{ExampleCode}
    gm.sim.ixj(1000,c(1,1,1),c(1,1,1,1,1))
    gm.sim.ixj(1000,c(.2,.3,.4,.1),c(.5,.2,.3)) 
    
    round(gm.sim.ixj(30,c(1,1),c(1,1)))       
    
    tab <- round(gm.sim.ixj(500,c(.5,.5),c(.5,.5)))
    chisq.test(tab)   
\end{ExampleCode}
\end{Examples}

\inputencoding{latin1}
\HeaderA{gmvalid-package}{Validation of graphical models}{gmvalid.Rdash.package}
\aliasA{gmvalid}{gmvalid-package}{gmvalid}
\keyword{datagen}{gmvalid-package}
\keyword{models}{gmvalid-package}
\keyword{multivariate}{gmvalid-package}
\keyword{nonparametric}{gmvalid-package}
\keyword{graphs}{gmvalid-package}
\keyword{htest}{gmvalid-package}
\begin{Description}\relax
This package provides functions among others that can be used to analyse
graphical models. This includes e.g. the possibility
to simulate data sets given a dependence model,
to analyze discrete graphical models utilizing the MIM program or the CoCo package
and to quantify associations or interactions.

Furthermore, a selected graphical model can be validated using the bootstrap and
the best prediction model can be evaluated for a dichotomous outcome variable and several discrete influences
using cross validation.
\end{Description}
\begin{Details}\relax
\Tabular{ll}{
Package: & gmvalid\\
Type: & Package\\
Version: & 1.0\\
Date: & 2007-11-07\\
License: & GPL (>=2) \\
}

\item This package requires the packages mimR and CoCo as well as the software MIM that can be downloaded at \url{http://www.hypergraph.dk}.
\item The mimR package itself requires the package Rgraphviz that is found in the Bioconductor repository.
\item Further the packages grid, epitools and gtools are required.
\item Before using any functions that use MIM make sure that the program is runnning.
\item The package mimR will only work properly if your Windows temporary directory has a path
where every folder has a name containing only 8 letters or less.
\item For mimR requirements we refer to the package's help page \code{\LinkA{mimR}{mimR}}.
\item If CoCo calculations fail, please type "endCoCo()" to remove temporary files. 
\item CoCo crashes occasionally. If you have recurring problems with a specific data set, we suggest to use the MIM-option. Otherwise reboot the Computer.
\item Model formulas have to be specified according to MIM model formulas or respectively CoCo model formulas.
\end{Details}
\begin{Note}\relax
This work has been supported by the German Research Foundation\\
(DFG: \url{http://www.dfg.de})
under grant scheme PI 345/2-1.
\end{Note}
\begin{Author}\relax
Ronja Foraita, Fabian Sobotka \\
Bremen Institute for Prevention Research and Social Medicine \\
(BIPS)  \url{http://www.bips.uni-bremen.de}
\end{Author}
\begin{References}\relax
> MIM (\url{http://www.hypergraph.dk/}) \\
Edwards D (2002) 
\emph{An Introduction to Graphical Modelling.}
Springer

> mimR (\url{http://genetics.agrsci.dk/~sorenh/mimR/index.html})\\
H�jsgaard S (2004) 
\emph{The mimR package for graphical modelling in R.}
Journal of Statistical Software, 11(6).

> CoCo (\url{http://www.badsberg.eu})\\
Badsberg JH (2001) 
\emph{A guide to CoCo.}
Journal of Statistical Software, 6(4).

> CSI \\
Foraita R (2007) 
\emph{A conditional synergy index to assess biological interaction.}
Working Paper. Please send an e-mail to \email{foraita@bips.uni-bremen.de}.
\end{References}
\begin{SeeAlso}\relax
\code{\LinkA{mimR}{mimR}},  \code{\LinkA{CoCo}{CoCo}}
\end{SeeAlso}
\begin{Examples}
\begin{ExampleCode}
    ### Generates a data frame given a dependence model
    gm.a <- gm.modelsim(1000,"ABC,CDE")
    
    ### Modelselection with graphical output
    gm.analysis(gm.a)   
    
    ### Model validation using the bootstrap 
    gm.boot.coco(100,gm.a,recursive=TRUE,follow=TRUE)

    ### Model prediction using cross validation
    gm.cv(3,data=gm.a,strategy="f",options="b")
    
    ### Testing interaction on the penetrance scale
    ### using the conditional synergy index (CSI)
    gm.csi(1,2,3,data=gm.a)

    ### Testing interaction on a additivity scale
    ### using the synergy index (S)
    gm.si(1,2,3,data=gm.a)   

    ### Gamma Coefficient B indpendent D given C
    gm.gamma(2,4,data=gm.a,conditions=3)

\end{ExampleCode}
\end{Examples}

\HeaderA{idd35}{Type 1 Diabetes susceptibility loci Idd3 and Idd5}{idd35}
\keyword{datasets}{idd35}
\begin{Description}\relax
Loci Idd3 and Idd5 are suspectable to be associated with type 1 diabetes.
This data reflects an animal experiment from Hill et al. (2000).
\end{Description}
\begin{Usage}
\begin{verbatim}data(idd35)\end{verbatim}
\end{Usage}
\begin{Format}\relax
A data frame with 313 observations on the following 3 variables.
\describe{
\item[\code{idd3}] a factor with levels \code{NN} and \code{BB}
\item[\code{idd5}] a factor with levels \code{NN} and \code{BB}
\item[\code{group}] a factor with levels \code{controls} \code{cases}
}
\end{Format}
\begin{Details}\relax
Data was used from Cordell et al. (2001) to model the joint effect of alleles at
the loci Idd3 and Idd5 on the outcome "type 1 diabetes". Only homozygeous genotypes 
were available.
\end{Details}
\begin{Source}\relax
Hill NJ, Lyons PA, Armitage N, Todd JA, Wicker LS and Peterson LB (2000)
\emph{NOD Idd5 locus controls insulitis and diabetes and overlaps the 
orthologous CTLA4/IDDM12 and NRAMP1 loci in humans.}
Diabetes, 49:1744-1747.
\end{Source}
\begin{References}\relax
Cordell HJ, Todd JA, Hill NJ, Lord CJ, Lyons PA, Peterson LB, Wicker LS and Clayton DG (2001)
\emph{Statistical modeling of interlocus interactions in a complex disease: Rejection
of the multiplicative model of epistasis in type 1 diabetes.}
Genetics, 158:357-367.
\end{References}
\begin{Examples}
\begin{ExampleCode}
  data(idd35)
  table(idd35)
\end{ExampleCode}
\end{Examples}

\HeaderA{wam}{Women and Mathematics}{wam}
\keyword{datasets}{wam}
\begin{Description}\relax
The data set describes the attitude of high school students toward
mathematics achievement and related topics.
\end{Description}
\begin{Usage}
\begin{verbatim}data(wam)\end{verbatim}
\end{Usage}
\begin{Format}\relax
A data frame with 1190 observations on the following 6 variables.
\describe{
\item[\code{lecture}] a factor with levels \code{1} (yes) \code{2} (no) 
\item[\code{gender}] a factor with levels \code{1} (female) \code{2} (male) 
\item[\code{school}] a factor with levels \code{1} (suburban) \code{2} (urban) 
\item[\code{work}] a factor with levels \code{1} (will need math) \code{2} (won't) 
\item[\code{subject}] a factor with levels \code{1} (science) \code{2} (arts) 
\item[\code{plans}] a factor with levels \code{1} (college) \code{2} (job) 
}
\end{Format}
\begin{Details}\relax
1190 high school students of eight schools in New Jersey responded to 
the questionnaire. The survey was performed by Lacampagne (1979).
\end{Details}
\begin{Source}\relax
Fowlkes, EB, Freeny, AE and Landwehr, JM (1988) \emph{Evaluating logistic models for large contingency tables.}
Journal of the American Statistical Association, 83:611-622.
\end{Source}
\begin{References}\relax
Lacampagne, CB (1979) \emph{An Evaluation of the Women and Mathematics (WAM) Program and Associated Sex-Related Differences in the Teaching, 
Learning and Counseling of Mathematics.}
Ed.D. Thesis, Columbia University Teachers College, USA.
\end{References}
\begin{Examples}
\begin{ExampleCode}
  data(wam)
  gm.analysis(wam, program="coco")
\end{ExampleCode}
\end{Examples}

\HeaderA{wynder}{Alcohol, Smoking and Oral Cancer}{wynder}
\keyword{datasets}{wynder}
\begin{Description}\relax
Alcohol drinking and smoking among cases of oral cancer and controls.
\end{Description}
\begin{Usage}
\begin{verbatim}data(wynder)\end{verbatim}
\end{Usage}
\begin{Format}\relax
A data frame with 443 observations on the following 3 variables.
\describe{
\item[\code{Alc}] a numeric vector indicating the units of alcohol.
\item[\code{Smoking}] a numeric vector indicating the units of smoked cigarettes.
\item[\code{Group}] a numeric vector indicating whether the subject is "control" (1) or "case" (2).
}
\end{Format}
\begin{Details}\relax
One unit of alcohol equals 1 oz (30 cc) of whiskey or 8 oz (237 cc) of beer.
\end{Details}
\begin{Source}\relax
Wynder EE, Bross IJ, Feldman R (1957) \emph{A study of etiological factors in cancer of the mouth.}
Cancer, 10:1300-1323.
\end{Source}
\begin{References}\relax
Rothman K (1974) \emph{The estimation of synergy or antagonism.}
American Journal of Epidemiology, 103(5):506-511
\end{References}
\begin{Examples}
\begin{ExampleCode}
  data(wynder)
  tab <- table(wynder)
  dimnames(tab) <- list(c("<1","1-6",">6"), c("<15","16-34",">34"), c("controls","cases"))  
  names(dimnames(tab)) <- c("Alcohol (unit/day)","Smoking (cigarettes/day)","Group")  
  tab
\end{ExampleCode}
\end{Examples}

\end{document}
