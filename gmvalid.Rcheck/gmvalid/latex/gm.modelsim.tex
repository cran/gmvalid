\HeaderA{gm.modelsim}{Random data frames with given dependence model and marginals}{gm.modelsim}
\keyword{datagen}{gm.modelsim}
\keyword{graphs}{gm.modelsim}
\begin{Description}\relax
Generates a random data frame of discrete variables with a given dependence model and marginals.
\end{Description}
\begin{Usage}
\begin{verbatim}
    gm.modelsim(N, model, categories = 0)
\end{verbatim}
\end{Usage}
\begin{Arguments}
\begin{ldescription}
\item[\code{N}] Number of observations, sample size. 
\item[\code{model}] A character string assigning a dependence model expressed as clique structure. Each variable
has to be expressed as a letter, e.g. "ABC,CDE".

\item[\code{categories}] a list of weight vectors that assigns the weight of each catogory.
Number of list elements must equal the number of variables in \code{model}.
Default is "list(c(.5,.5),c(.5,.5),...)".

\end{ldescription}
\end{Arguments}
\begin{Value}
A data frame with number of rows approximately equal to \code{N} and number of columns equal
to the number of variables used in \code{model}.
\end{Value}
\begin{Note}\relax
Observed marginal probabilities reflect the given marginal probabilites only approximatively.
Works best with population sizes over N=10,000.
\end{Note}
\begin{Author}\relax
Ronja Foraita, Fabian Sobotka \\
Bremen Institute for Prevention Research and Social Medicine \\
(BIPS)  \url{http://www.bips.uni-bremen.de}
\end{Author}
\begin{SeeAlso}\relax
\code{\LinkA{gm.generate}{gm.generate}}, \code{\LinkA{gm.sim.ixj}{gm.sim.ixj}}, \code{\LinkA{r2dtable}{r2dtable}}
\end{SeeAlso}
\begin{Examples}
\begin{ExampleCode}
    gm.modelsim(100,"AB,AC")
    table( gm.modelsim(100,"a,b,c") )
    
    tmp.df <- gm.modelsim(10000,"abf,cd,cf,bdeg,bfg")
    
    # with given number of categories
    tmp.df <- gm.modelsim(1000,"AB,C",list(c(1,1,1),c(1,1),c(1,1,1)))

    # with given number of categories and marginals
    tmp.df <- gm.modelsim(1000,"ABC",list(c(0.3,0.3,0.4),c(0.6,.4),c(0.25,0.25,0.5)))
    table(tmp.df)

    ## Not run: 
tmp.df <- gm.modelsim(100,"ABC",list(3,2,3))# (number of categories will be 2 x 2 x 2 )
            gm.modelsim(100,"123")
            
## End(Not run)
\end{ExampleCode}
\end{Examples}

