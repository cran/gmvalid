\HeaderA{gm.gamma}{Conditional Gamma coefficient estimation and confidence intervals}{gm.gamma}
\keyword{htest}{gm.gamma}
\begin{Description}\relax
Calculates conditional gamma coefficients of two binary or ordinal variables X and Y given a 
set of variables \{A,B,...\}.
\end{Description}
\begin{Usage}
\begin{verbatim}
  gm.gamma(X = 0, Y = 0, data = 0, conditions = 0,
        type = c("conditional", "single", "marginal"), conf.level = 0.95)
\end{verbatim}
\end{Usage}
\begin{Arguments}
\begin{ldescription}
\item[\code{X}] Index of the variable's position in \code{data} or a vector. 
\item[\code{Y}] Index of the variable's position in \code{data} or a vector. 
\item[\code{data}] Data frame or table 
\item[\code{conditions}] Variable indices in \code{data} or a data frame of the conditioning variables. 
\item[\code{type}] Character string specifying the method,
must be one of "conditional" (default), "single" or "marginal". 
May be abbreviated. 
\item[\code{conf.level}] Confidence level of the interval (default is 0.95). 
\end{ldescription}
\end{Arguments}
\begin{Details}\relax
The gamma coefficient is a correlation measure between two
binary or ordinal variables. It ranges between -1 and 1, where -1 or respectively
+1 stands for a purely negative or positive monotone relation. The relation has
not to be of linear nature!

If \code{type} = "conditional", the conditional gamma coefficients are calculated and
if \code{type} = "marginal", the marginal gamma coefficients are computed. \\
If \code{X} and \code{Y} are given, the "single" gamma coefficient between both
variables are computed.\\
If \code{X} or \code{Y} are zero, the function computes all possible conditional gamma coefficients.

Confidence intervals are calculated using the asymptotic variance given in Olszak and Ritschard (1995).
\end{Details}
\begin{Value}
A matrix containing the gamma estimate(s), standard deviation(s), confidence interval(s) and p-value(s).
\end{Value}
\begin{Author}\relax
Ronja Foraita, Fabian Sobotka \\
Bremen Institute for Prevention Research and Social Medicine \\
(BIPS)  \url{http://www.bips.uni-bremen.de}
\end{Author}
\begin{References}\relax
Davis JA (1967) 
\emph{A partial coefficient for Goodman and Kruskal's gamma.}
Journal of the American Statistical Association, 62:189-193.

Olszak M, Ritschard G (1995) 
\emph{The behaviour of nominal and ordinal partial association measures.}
The Statistician, 44(2):195-212.
\end{References}
\begin{SeeAlso}\relax
\code{\LinkA{gm.or}{gm.or}}, \code{\LinkA{gm.rr}{gm.rr}}
\end{SeeAlso}
\begin{Examples}
\begin{ExampleCode}
  data(dp)

  ### Conditional Gamma by victime
  gm.gamma(1,3,conditions=2,data=dp)
  ### the same
  gm.gamma(dp$Defendants.Race,dp$Death.Penalty,data=dp,conditions=dp$Victims.Race)
  
  ### Stratified Gamma
  dp.black <- data.frame(victime=dp$Victims.Race[dp$Victims.Race=="black"],
                        killer=dp$Defendants.Race[dp$Victims.Race=="black"],
                        death.penalty=dp$Death.Penalty[dp$Victims.Race=="black"])
  dp.white <- data.frame(victime=dp$Victims.Race[dp$Victims.Race=="white"],
                        killer=dp$Defendants.Race[dp$Victims.Race=="white"],
                        death.penalty=dp$Death.Penalty[dp$Victims.Race=="white"])  
  table(dp.black[,c(2,3,1)])
  table(dp.white[,c(2,3,1)])  

  gm.gamma(2,3,data=dp.black)  
  gm.gamma(2,3,data=dp.white)  
  
  ### Marginal Gamma
  gm.gamma(1,3,data=dp)

  ### Analyse complete data set
  gm.gamma(data=dp,type="m")
  
  ### Plot model
  gamma <- gm.gamma(data=dp)
   #> all edges
  mat <- matrix(NA,nrow=3,ncol=3)
  mat[upper.tri(mat)] <- gamma[,1]
  gm.plot(model="abc",data.analysis=mat)
   #> only significant edges
  mat <- matrix(NA,nrow=3,ncol=3)   
  tmp <- vector()
  for( i in 1:dim(gamma)[1] ) ifelse(gamma[i,5]<0.05, tmp[i] <- gamma[i,1], tmp[i] <-NA)
  mat[upper.tri(mat)] <- tmp
  gm.plot(model="ab,bc",data.analysis=mat)
\end{ExampleCode}
\end{Examples}

