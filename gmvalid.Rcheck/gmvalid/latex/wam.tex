\HeaderA{wam}{Women and Mathematics}{wam}
\keyword{datasets}{wam}
\begin{Description}\relax
The data set describes the attitude of high school students toward
mathematics achievement and related topics.
\end{Description}
\begin{Usage}
\begin{verbatim}data(wam)\end{verbatim}
\end{Usage}
\begin{Format}\relax
A data frame with 1190 observations on the following 6 variables.
\describe{
\item[\code{lecture}] a factor with levels \code{1} (yes) \code{2} (no) 
\item[\code{gender}] a factor with levels \code{1} (female) \code{2} (male) 
\item[\code{school}] a factor with levels \code{1} (suburban) \code{2} (urban) 
\item[\code{work}] a factor with levels \code{1} (will need math) \code{2} (won't) 
\item[\code{subject}] a factor with levels \code{1} (science) \code{2} (arts) 
\item[\code{plans}] a factor with levels \code{1} (college) \code{2} (job) 
}
\end{Format}
\begin{Details}\relax
1190 high school students of eight schools in New Jersey responded to 
the questionnaire. The survey was performed by Lacampagne (1979).
\end{Details}
\begin{Source}\relax
Fowlkes, EB, Freeny, AE and Landwehr, JM (1988) \emph{Evaluating logistic models for large contingency tables.}
Journal of the American Statistical Association, 83:611-622.
\end{Source}
\begin{References}\relax
Lacampagne, CB (1979) \emph{An Evaluation of the Women and Mathematics (WAM) Program and Associated Sex-Related Differences in the Teaching, 
Learning and Counseling of Mathematics.}
Ed.D. Thesis, Columbia University Teachers College, USA.
\end{References}
\begin{Examples}
\begin{ExampleCode}
  data(wam)
  gm.analysis(wam, program="coco")
\end{ExampleCode}
\end{Examples}

