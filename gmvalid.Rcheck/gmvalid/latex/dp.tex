\HeaderA{dp}{Death penalty example of Simpson's paradox}{dp}
\keyword{datasets}{dp}
\begin{Description}\relax
Well known example to illustrate Simpson's paradox.
The data set shows that ignoring victim's race lead to a different conclusion
than including victim's race in the analysis.
\end{Description}
\begin{Usage}
\begin{verbatim}data(dp)\end{verbatim}
\end{Usage}
\begin{Format}\relax
A data frame with 326 observations on the following 3 variables.
\describe{
\item[\code{Defendants.Race}] a factor with levels \code{white} and \code{black}
\item[\code{Victims.Race}] a factor with levels \code{white} and \code{black}
\item[\code{Death.Penalty}] a factor with levels \code{yes} and \code{no}
}
\end{Format}
\begin{Source}\relax
Radelet ML (1981) \emph{Racial characteristics and the imposition of the Death penalty.}
American Sociological Review, 46(6):918-927.
\end{Source}
\begin{Examples}
\begin{ExampleCode}
data(dp)
## Graphical model analysis shows that 'defendants' race' is 
## independent from 'death penalty' given 'victims' race'.
gm.analysis(dp,program="coco",recursive=TRUE)
\end{ExampleCode}
\end{Examples}

