\HeaderA{gm.csi}{Conditional Synergy Index}{gm.csi}
\keyword{htest}{gm.csi}
\begin{Description}\relax
Computes the conditional synergy index of two dichotomous variables in relation to 
a binary group variable on the penetrance scale.
\end{Description}
\begin{Usage}
\begin{verbatim}
gm.csi(X, Y, group, data=0, reference=c(1,1,2), pen = NULL, conf.level = 0.95)
\end{verbatim}
\end{Usage}
\begin{Arguments}
\begin{ldescription}
\item[\code{X}] Index of the variable's position in \code{data} or a vector. 
\item[\code{Y}] Index of the variable's position in \code{data} or a vector. 
\item[\code{group}] Binary group or outcome variable addressed as variable index in 
\code{data} or as vector. By default, 2 is the reference category.
\item[\code{data}] Data frame or table. 
\item[\code{reference}] Vector to define the \code{reference} categories of \code{X}, \code{Y} and \code{group}. 
By default, 2 is the reference category for group.
\item[\code{pen}] If FALSE the penetrance P( group | X,Y ) is estimated from the data. 
\item[\code{conf.level}] Confidence level of the interval (default is 0.95). 
\end{ldescription}
\end{Arguments}
\begin{Details}\relax
Foraita's conditional synergy index (CSI) is an interaction measure on the penetrance scale
between two dichotomous variables on a  binary outcome. 
The index equals 1 under additivity, CSI > 1 in the case of antagonism and 
CSI < 1 in the presence of synergy.

The confidence intervals are calculated using an asymptotic variance given in Foraita (2007).
\end{Details}
\begin{Value}
A list containing:
\begin{ldescription}
\item[\code{penetrance.ratio}] a penetrance ratio table of P( group=reference | X,Y ) / P( group=NOT reference| X,Y ) 
\item[\code{measure}] a matrix containing the estimate, standard deviation, 
confidence interval and p-value. Figures in brackets show the
reference category respectively the category under consideration.
\end{ldescription}
\end{Value}
\begin{Author}\relax
Ronja Foraita, Fabian Sobotka \\
Bremen Institute for Prevention Research and Social Medicine \\
(BIPS)  \url{http://www.bips.uni-bremen.de}
\end{Author}
\begin{References}\relax
Foraita R (2007)
\emph{A conditional synergy index to assess biological interaction.}
Working Paper. Please send an e-mail to \email{foraita@bips.uni-bremen.de}.
\end{References}
\begin{SeeAlso}\relax
\code{\LinkA{gm.si}{gm.si}}
\end{SeeAlso}
\begin{Examples}
\begin{ExampleCode}
  data(idd35)
  gm.csi(1,2,3,data=idd35)

  ### >> constructing an additive and multiplicative penetrance
  x <- c(0.1,0.4)
  y <- c(0.05,0.5)
  add.pen <- outer(x,y,FUN="+")
  mult.pen <- outer(x,y)
  het.pen <- outer(x,y,FUN="+") - outer(x,y)

  ### >> Function that samples data using the penetrance 
  make.data <- function(R,pen,category) 
    {
      s.vec <- sample(c(1,2,3,4),R,replace=TRUE,prob=as.vector(pen))
      fact.1 <- fact.2 <- vector()
      for( i in 1:R ) {
        ifelse( s.vec[i] == 1 || s.vec[i] == 3 , fact.1[i] <- 1, fact.1[i] <- 2 ) 
        ifelse( s.vec[i] == 1 || s.vec[i] == 2 , fact.2[i] <- 1, fact.2[i] <- 2 ) 
      }
      cbind(X=fact.1,Y=fact.2,group=rep(category,R))  
    }

  ### >>> Building datasets with affected and unaffected subjects   
  add.aff <- make.data(200,add.pen,2)
  add.uaf <- make.data(200,1-add.pen,1)  
  add.df <- as.data.frame(rbind(add.uaf,add.aff))
  
  mult.aff <- make.data(200,mult.pen,2)
  mult.uaf <- make.data(200,1-mult.pen,1)  
  mult.df <- as.data.frame(rbind(mult.uaf,mult.aff))
  
  het.aff <- make.data(200,het.pen,2)
  het.uaf <- make.data(200,1-het.pen,1)  
  het.df <- as.data.frame(rbind(het.uaf,het.aff))
   
  gm.csi(1,2,3,add.df,pen=add.pen)   # Additivity
  gm.csi(1,2,3,mult.df,pen=mult.pen) # Synergy
  gm.csi(1,2,3,het.df,pen=het.pen)   # Antagonism
\end{ExampleCode}
\end{Examples}

