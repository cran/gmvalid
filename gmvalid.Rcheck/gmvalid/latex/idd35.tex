\HeaderA{idd35}{Type 1 Diabetes susceptibility loci Idd3 and Idd5}{idd35}
\keyword{datasets}{idd35}
\begin{Description}\relax
Loci Idd3 and Idd5 are suspectable to be associated with type 1 diabetes.
This data reflects an animal experiment from Hill et al. (2000).
\end{Description}
\begin{Usage}
\begin{verbatim}data(idd35)\end{verbatim}
\end{Usage}
\begin{Format}\relax
A data frame with 313 observations on the following 3 variables.
\describe{
\item[\code{idd3}] a factor with levels \code{NN} and \code{BB}
\item[\code{idd5}] a factor with levels \code{NN} and \code{BB}
\item[\code{group}] a factor with levels \code{controls} \code{cases}
}
\end{Format}
\begin{Details}\relax
Data was used from Cordell et al. (2001) to model the joint effect of alleles at
the loci Idd3 and Idd5 on the outcome "type 1 diabetes". Only homozygeous genotypes 
were available.
\end{Details}
\begin{Source}\relax
Hill NJ, Lyons PA, Armitage N, Todd JA, Wicker LS and Peterson LB (2000)
\emph{NOD Idd5 locus controls insulitis and diabetes and overlaps the 
orthologous CTLA4/IDDM12 and NRAMP1 loci in humans.}
Diabetes, 49:1744-1747.
\end{Source}
\begin{References}\relax
Cordell HJ, Todd JA, Hill NJ, Lord CJ, Lyons PA, Peterson LB, Wicker LS and Clayton DG (2001)
\emph{Statistical modeling of interlocus interactions in a complex disease: Rejection
of the multiplicative model of epistasis in type 1 diabetes.}
Genetics, 158:357-367.
\end{References}
\begin{Examples}
\begin{ExampleCode}
  data(idd35)
  table(idd35)
\end{ExampleCode}
\end{Examples}

