\HeaderA{gm.si}{Synergy Index}{gm.si}
\keyword{htest}{gm.si}
\begin{Description}\relax
Computes the synergy index of two discrete variables in relation to 
a binary group variable.
\end{Description}
\begin{Usage}
\begin{verbatim}
gm.si(X,Y,group,data=0,reference=c(1,1,2),conf.level = 0.95)
\end{verbatim}
\end{Usage}
\begin{Arguments}
\begin{ldescription}
\item[\code{X}] Index of the variable's position in \code{data} or a vector. 
\item[\code{Y}] Index of the variable's position in \code{data} or a vector. 
\item[\code{group}] Binary group or outcome variable addressed as variable index in 
\code{data} or as vector.
\item[\code{data}] Data frame or table. 
\item[\code{reference}] Vector to define the \code{reference} categories of \code{X}, \code{Y} and \code{group}.
By default, 2 is the reference category for group.
\item[\code{conf.level}] Confidence level of the interval (default is 0.95). 
\end{ldescription}
\end{Arguments}
\begin{Details}\relax
Rothman's synergy index (S) is an interaction measure between two
discrete variables on a dichotomous outcome. 
The index equals 1 under additivity, S > 1 in the case of synergy and 
S < 1 in the presence of antagonism.

The synergy index is originally constructed on (2 x 2)-tables, but (i x j)-tables can
also be analyzed. Then (i-1) x (j-1) synergy indices are computed and combined to 
an overall synergy index.    

The confidence intervalls are calculated using the asymptotic variance given in 
Rothman (1974).
\end{Details}
\begin{Value}
A list containing:
\begin{ldescription}
\item[\code{ratio }] A rate ratio table, more precise [P( group=reference,X,Y ) / P( group=NOT reference,X,Y )] / [min(P( group=reference,X,Y ) / P( group=NOT reference,X,Y ))] 
\item[\code{covariance}] Covariances matrix of the single synergy indices. 
Not written if \code{X} or \code{Y} are binary.
\item[\code{measure }] Matrix containing the estimate(s), standard deviation(s), 
confidence interval(s) and p-value(s). Figures in brackets show the
reference category respectively the category under consideration.\\                 
If both factors \code{X} and \code{Y} are binary, confidence 
intervals for case-control as well as cohort designs are computed.
If at least one factor has more than 2 categories, the overall synergy index
with its corresponding confidence interval is computed that follows a 
case-control design.
\end{ldescription}
\end{Value}
\begin{Note}\relax
It can occur that certain combinations of categories lead to a negative synergy indices.
In that case no confidence intervals can be computed. If so, use the \code{reference} option
to re-order the categories of the variable(s) in question (see example below).
\end{Note}
\begin{Author}\relax
Ronja Foraita, Fabian Sobotka \\
Bremen Institute for Prevention Research and Social Medicine \\
(BIPS)  \url{http://www.bips.uni-bremen.de}
\end{Author}
\begin{References}\relax
Rothman K (1974) 
\emph{The estimation of synergy or antagonism.}
American Journal of Epidemiology, 103(5):506-511

Rothman K (1986) 
\emph{Modern Epidemiology.}
Little, Brown and Company, Boston/Toronto.
\end{References}
\begin{SeeAlso}\relax
\code{\LinkA{gm.csi}{gm.csi}}
\end{SeeAlso}
\begin{Examples}
\begin{ExampleCode}
  data(wynder)
  gm.si(1,2,3,wynder)

  # Smoking and alcohol in relation to oral cancer among male veterans under age 60.
  # (from "Modern Epidemiology")
  oral <- array(c(20,3,18,8,12,6,166,225),dim=c(2,2,2), 
            dimnames=list(Group=c("control","cases"),
            Smoker=c("no","yes"),Alcohol=c("no","yes")))
  oral.df <- expand.table(oral)
  # grouping variable is first in data frame
  gm.si(2,3,1,oral.df)
  
  # Effects must be ascending in respect to the reference category
  show.effect <- array(c(1,7,2,7,7,12,106,48),dim=c(2,2,2),
                        dimnames=list(A=1:2,B=1:2,C=1:2))
  # produces NaN
  gm.si(1,2,3,expand.table(show.effect))
  # > re-ordering variable B helps
  gm.si(1,2,3,expand.table(show.effect),reference=c(1,2,2))

\end{ExampleCode}
\end{Examples}

